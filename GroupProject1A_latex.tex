\documentclass[a4paper,12pt]{article}
\usepackage{enumerate,amsmath,graphicx,float,epstopdf}
\graphicspath{ {./} }
\begin{document}

\title{PHY566 Group Project 1 (A) \\ Random Walk, Diffusion, Cluster Growth}
\date{\small April 3rd, 2015}
\author{David Hicks\\ Weiyao Ke \\ Shagun Maheshwari \\ Fan Zhang}

\maketitle

\section{2D Random Walk}
.............


\section{Diffusion Equation}
\subsection{1D diffusion Equation Outline}
\indent
\indent The homogeneous 1D diffusion Equation reads,
\begin{equation}
	\frac{\partial f}{\partial t} = D \nabla^2 f.
\end{equation}
where $D$ is the diffusion constant. For a initial delta distribution $f(x, t = 0) = \delta(x)$, the time evolution of the diffusing density profile is solved by a expanding gaussian wave packet,

\begin{eqnarray*}
	f(x,t) &=& \frac{1}{\sqrt{2\pi D t}} \exp{(-\frac{x^2}{2Dt})}, \\
	f(x, t \rightarrow 0^+) &=& \delta(x).
\end{eqnarray*}

Generally, the expectation value $<x^2(t)>$ at certain time $t$ of a gaussian distribution with a time dependent width $\sigma(t)$ is,
\begin{eqnarray*}
	<x^2(t)> = \int_{-\infty}^{+\infty} x^2 f(x,t) dx &=&  \int_{-\infty}^{\infty}  \frac{1}{\sqrt{2\pi \sigma^2(t)}} x^2 \exp{(-\frac{x^2}{2\sigma^2(t)})} dx, \\
	& = & -\lim_{\beta\rightarrow 1}\frac{d}{d\beta}\int_{-\infty}^{\infty}  \sqrt{\frac{\sigma^2(t)}{\pi}} \exp{(-\beta \frac{x^2}{2\sigma^2(t)})} dx, \\
	& = & -\lim_{\beta\rightarrow 1} 2\sigma^2(t)\frac{d}{d\beta}\sqrt{\beta}, \\
	& = & \sigma^2(t). 
\end{eqnarray*}
As a result, we expect that the root mean square of the solution of the diffusion equation with an delta function like initial configuration grows like $Dt$ with the time evolution. 

To verify this argument above, we solve the 1D diffusion equation numerically, fit the solution at several time step by gaussian wave packet and extract the time evolution of the root mean square $<x^2>$ as a function of $t$.

\subsection{Algorithm}
To solve the 1D diffusion numerical, we first discretise the partial differential with respect to time and space:
\begin{eqnarray*}
\frac{f_{n+1,i}-f_{n,i}}{\Delta t}  &=& D\frac{f_{n,(i+1)\mod N}+f_{n,(i-1)\mod N} - 2 f_{n,i}}{{\Delta x}^2},\\
t &=& n \Delta t, \\
x &=& i \Delta x.
\end{eqnarray*}
$N$ is the number of spatial grids, the $\mod$ function imposes a periodic boundary condition of the problem, which is equivalent to solve the diffusion equation on a 1D ring.

This is then cast into a time evolution form of a time iteration:
\begin{eqnarray*}
f_{n+1,i} &=&  (1 - 2 \alpha) f_{n,i} + \alpha (f_{n, (i+1)\mod N}+f_{n, (i-1)\mod N}), \alpha = \frac{D\Delta t}{{\Delta x}^2},\\
f_{0,j} &=& f_0(j\Delta x).
\end{eqnarray*}
where $f_0$ is the initial density distribution. Here we set $f_0$ zero except for a short interval resembling a delta function initial condition. The stability criterion for this method is given by $\alpha < 1/2$. This condition is always checked ahead of calling the solver. 

\subsection{Results and Discussion}


\section{Cluster Growth with the DLA Model}
\subsection{Problem Outline}
\indent
\indent The random walk concept can also be used to model cluster growth.  In particular, there are two common types of cluster growth methods involving random walks.
The first is known as the Eden Model.  In the Eden model, a seed particle is place in the center of our system and the all of the perimeter sites are identified.  The 
random walk concept is used to randomly choose at which perimeter site a particle will be added.  This is process continues until the desired cluster size is attained.
In this method, the cluster is typically compact with a few number of holes due to the nature of this growth model.  The Eden model grows clusters from the "center-out" 
and is referred to as a "cancer model". 

\indent The other common cluster growth is the diffusion limited aggregation (DLA) model.  Similar to the Eden model, the DLA model starts with a seep particle at 
the center of our system; however, the added particle is randomly generated at a certain distance from the center of the system.  The particle must then perform 
numerous random walks until the particle hits one of the perimeter sites of the central cluster.  The perimeter sites are updated and the procedure continues until
the cluster is a specified size.  In contrast to the Eden model, the DLA model clusters are often extremely sparse and consist of many branches stemming from the 
central cluster.

\subsection{Solution}
\indent
\indent The growth of a cluster was simulated using the DLA model.  The proposed radius of the system is about 100 units.  A seed particle was placed in the center of 
the system and for each iteration the intial position of a random walker at a radius of 100 was generated.  The random walk of this walker was determined using a 
random number generator.  The directions up, down, left, and right were chosen depending on the random number between 0 and 1 (see Python code). The walker's random path
continued until it hit the cluster's perimeter or until it traveled a certain radius away from the cluster.  If the walker traveled a certain distance away from the 
cluster then the walker is said to not return to the cluster in a sufficiently short amount of time, so the walker is thrown out and a new one is generated.

\indent The aformentioned procedure was carried out until the desired radius of 100 units was obtained.  A total of 10 clusters were grown using the DLA method.  The
results are shown in the next section.

\subsection{Results}  
\indent RESULTS TO COME....



\end{document}
